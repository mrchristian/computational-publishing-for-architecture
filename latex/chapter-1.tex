% © 2022 Simon Worthington
% https://creativecommons.org/licenses/by-sa/4.0/


\documentclass{article}

                
\usepackage{titling}
                
\newcommand{\subtitle}[1]{%
                    
	\posttitle{%
                        
		\par\end{center}
                        
		\begin{center}\large#1\end{center}
                        
		\vskip 0.5em}%
                }
            
                
\usepackage{authblk}
                
\makeatletter
                
\let\@fnsymbol\@alph
                
\makeatother
            
                
\def\keywords{\vspace{.5em}
                
{\textit{Keywords}:\,\relax%
                
}}
                
\def\endkeywords{\par}
                
\newcommand{\sep}{, }
            
\usepackage{hyperref}
                
\usepackage[backend=biber,hyperref=false,citestyle=authoryear,bibstyle=authoryear]{biblatex}
                
\bibliography{bibliography}
            
\usepackage{enumitem}
\author{Simon Worthington\thanks{simon.worthington@tib.eu}}
\affil{NextGen Books – making the future book, TIB }


\begin{document}

\title{Computational Publishing for Architecture}
\subtitle{ — An X-Sketchbook Research Work Plan}

\maketitle

\begin{abstract}


The work plan is for an experimental publishing prototype as part of the X-Sketchbook project exploring digital publishing for architecture. The project is organised by TIB Open Science Lab and the Bartlett School. The aim of the work programme is to produce a small proof of concept publication containing a set of sample digital objects. It is being conducted in the context of NFDI4Culture — the German Consortium for Research Data on Material and Immaterial Cultural Heritage in collaboration with COPIM the Community-Led Open Publication Infrastructures for Monographs project, an international research partnership working on open infrastructures for monographs.


The central question for the project is how can an existing publisher's infrastructures and workflows incorporate computational publishing — the combinations of text and executable code — as applicable to the topic area of architecture. 


In addition we will be looking at how 'enhanced publication principles' for open access and FAIR‑publishing can be applied to  computational publishing. 


\#XSketchbook – \href{https://github.com/TIBHannover/xsketch}{https://github.com/TIBHannover/xsketch}


Keywords:

\end{abstract}

\keywords{computational publishing\sep enhanced publishing\sep open access\sep open science publishing\sep academic publishing\sep infrastructure\sep NFDI4Culture\sep NFDI\sep architecture\sep open source\sep open standards\sep linked open data\sep publishing from archives}

Date: 2022-05-10 


© The Authors, Creative Commons: Attribution-ShareAlike 4.0 International (CC BY SA 4.0) \href{https://creativecommons.org/licenses/by-sa/4.0/}{https://creativecommons.org/licenses/by-sa/4.0/} 

\begin{itemize}
\item \href{https://nfdi4culture.de/}{NFDI4Culture} – Consortium for Research Data on Material and Immaterial Cultural Heritage. NFDI4Culture is the consortium within the Nationale Forschungsdaten­infrastruktur (NFDI).


\item \href{https://www.copim.ac.uk/}{COPIM} – Community-led Open Publication Infrastructures for Monographs.


\end{itemize}

\subsection{Working definitions}\label{H6504202}


\begin{itemize}
\item \textbf{Enhanced publication principles for open access and FAIR‑publishing}\textbf{:} For enhanced publications the question is how to make a \textbf{publication open} in the context of modern computational and networked systems, and what \textbf{additional functionality} and \textbf{enrichment} can be added. What types of \textbf{requirements }are made of scholarly publishing to be fully open, and \textbf{how these are implemented }in systems or infrastructures. Example features are: PIDs, expanding roles and attribution, linking outputs, etc. The Enhanced Publications project is an activity of NFDI4Culture, Task Area 4. A working paper will be published in June 2022 for consultation on the implementation of enhanced publications. Contributors are: Alexandra Büttner, Ruprecht-Karls-Universität Heidelberg; Matthias Arnold, Ruprecht-Karls-Universität Heidelberg; Jörg Heseler, Sächsische Landesbibliothek – Staats- und Universitätsbibliothek Dresden, and; Simon Worthington, TIB – Leibniz-Informationszentrum Technik und Naturwissenschaften.


\item \textbf{Computational publishing}\textbf{ (Literate programming): }\autocite{KnuthDonald1992} Computational publishing means to combine both text and advanced computational functionality on a single hypertext document. Jupyter Notebook is one such tool where code, data, and findings can be presented in one instance for speedy verification (the \emph{replication crisis} in science literature is one reason for Jupiter Notebook's current popularity).


Computational publishing is not new in computing, in ideas of knowledge management, or in the history or imaginations of book technologists and thinkers. In modern computing Alan Kay articulated it as the \emph{Dynabook} \autocite{OdewahnAndrew20210619T14:43:08Z} in his paper \href{https://dl.acm.org/doi/abs/10.1145/800193.1971922?msclkid=2aa89ce0d03c11ec81b1043fb5e37b40}{A Personal Computer for Children of All Ages} \autocite{KayAlan1972} and went on to implement the concept, that same year, in the programming language for children's education called \href{https://en.wikipedia.org/wiki/Smalltalk}{Smalltalk} which allowed the editing of parameters to control the behaviour of objects on a screen for teaching concepts in geometry, music, literature, or maths. Kay added an interactive layer to existing ideas of networked knowledge publishing systems, and extensively acknowledges Douglas Engelbart's '\href{https://www.youtube.com/watch?v=yJDv-zdhzMY}{The Mother of All Demos}' \autocite{EngelbertDouglas1968}. Which in turn can be seen as the idea of the \emph{Memex} introduced in the 1945 Vannevar Bush essay \href{https://en.wikipedia.org/wiki/As_We_May_Think}{As We May Think} from \emph{The Atlantic} magazine \autocite{BushVannevar19450701} and then realised by Engelbert as the first networked desktop digital workstation. Engelbart and Bush do not explicitly propose computational publishing, but instead are more focused on ideas of networked knowledge and new interfaces based on their available 'knowledge galaxies'. Earlier analog, pre-computing, networked knowledge systems are for example Paul Otlet's 1934 \emph{\href{https://archive.org/details/OtletTraitDocumentationUgent/mode/2up?msclkid=329a575ad03a11ec870080b223f9d3a0}{Traité de Documentation}} \autocite{OtletPaul1934} which describes a paper based and telegraphic systems that we're put in place for cataloguing part of publications, 'Inspired by the arrival of radio, phonograph, cinema, and television' \autocite{ReagleJoseph2012}. It has to be noted that Otlet's project sits under a perverse context that was heavily contested even at the time — on the one hand the project was part of a utopian vision of world peace and equality and closely aligned to the formation of the League of Nations, while at the same time directly funded from the profits from the profits of King Leopold II of Belgium's brutal regime in the Congo Free State. Mark Twain wrote at the time, the famous pamphlet \emph{\href{https://archive.org/details/kingleopoldssoli00twaiuoft/page/n1/mode/2up}{King Leopold's Soliloquy}} \autocite{TwainMark1905} which grapically documented the killling and torture in the Congo.  


For this project the following computational publishing tools will be used: Jupyter Notebook, Jupyter Book, and Curvenote.


\item \textbf{Architecture:} We are focusing on architecture as it is the specialist subject of the TIB Library as well an area of research that TIB is contributing to for the NFDI4Culure research programme. NFDI4Culture concerns predominantly relate to the history of architecture and its preservation in digital heritage archives such as \href{http://www.urbanhistory4d.org/wordpress/}{Research Group Urban History 4D} and historical buildings of Dresden. 


In relation to contemporary practice architecture this is being explored in terms of modern studio practice through the \href{https://tibhannover.github.io/xsketch/}{X-Sketchbook} experimental publishing project which is looking at the architects 'sketchbook' in studio practice and the issues of making a digital skectbook that can record and catalogue heterogeneous digital objects generated from 3D platforms like \href{https://sketchfab.com/}{Sketchfab} or game engines like \href{https://unity.com/}{Unity}, to objects taken from social media, to robotics, sensors, or AI assisted CNC modelling.  


\item \textbf{Publication infrastructures:} For open publishing using computational publishing and enhanced publishing principles the main questions are how computational books can be integrated and made compatible with existing publishing infrastructures and workflows. The publishing infrastructures being covered will be those that apply to conventional monographs and research papers. The enhanced publication features would be, for example: real-time collaborative editing platforms, linked open data, 3D models, and semantic video, etc. Some of the infrastructures that we will want to integrate or test with the proof-of-concepts will be: 

\begin{itemize}
\item \href{https://github.com/OpenBookPublishers}{Open Book Publishers} (OBP) COPIM member – conventional open-source publishing infrastructure.


\item \href{https://github.com/TIBHannover/ADA}{ADA Pipeline} – Multi-format publishing pipeline from the Open Science Lab, TIB – The ADA Pipeline supports the TIB Service – \href{https://projects.tib.eu/nextgen-books/en/}{NextGen Books}.


\item \href{https://av.tib.eu/ }{TIB AV Portal} – Scholarly semantic video repository and open-source toolkit from TIB


\item \href{https://mvp.enrich-nfdi4culture.wiki/wiki/Main_Page}{MVP Semantic annotation for 3D cultural artefacts} – WikiData / \href{https://wikiba.se/}{Wikibase} / \href{https://kompakkt.de/home}{Kompakkt} / \href{https://openrefine.org/}{OpenRefine} – Team NFDI4Culture TA 1 (TIB) – \href{https://drive.google.com/file/d/1Umn2JS_-GEQONIdJ_ttlThl1VkwUwpae/view}{Demonstration video} 


\item \href{https://thoth.pub/}{Thoth} – COPIM partner. Book metadata platform


\item Unity – Game engine and \href{https://unity.com/solutions/architecture-engineering-construction}{3D Software for Architecture, Engineering \& Construction} 


\end{itemize}

\end{itemize}

\subsection{Motivation}\label{H2831995}



The utopian vision of computational publishing has inspired us with its promise of a better world through the use of universally interconnected knowledge and learning, and how this might potentially be modelled in forms of digital publishing. Work at PARC (Palo Alto Research Center) and on Kay's Dynabook was taken and made into the products of the personal computer and later the tablet computer. But at the same time, in its more than seventy-five years history, computational publishing itself, as a vision and paradigm, has failed to be realised. 


The recent prominence of Jupyter Notebook has shown a promising route for exploring computational publishing further as it offers a substantial and flexible publishing framework that a large number of stakeholders have bought into. Yet applying this to a 'traditional' digital and print book publishing workflow will be challenging, on both technological and socio-cultural fronts. This is one of the aspects we would like to explore in this project.


Architecture offers exciting opportunities for computational publishing and many computational features are already being explored within the field. These include: data visualisations and simulations, for the manipulation of design tools for robotics such as component fabrication, and in the presentation and exchange of ideas in 3D multi-modal-media and on social media platforms.


The following 2019 conference video \href{https://www.youtube.com/watch?v=rs1rNLetobY}{Ubiquity and Autonomy} from the Association for Computer Aided Design in Architecture gives a clear idea of modern design challenges for architecture.


Enhanced Publishing principles and testing implementation plans can play an important role in contributing to the sustainability and to real-world working models of computational publishing to support its use on monographs and research publications.


\subsection{Work plan}\label{H819867}


\begin{quote}



The overall purpose of the work plan is to produce \emph{a sample book} for further research of the types of computational objects that could be used in architecture and open up questions for publishers and publishing technologists.


\end{quote}

\begin{enumerate}
\item Set up work plan with the communities involved: NFDI4Culture TA4 Data Publishing, COPIM, TIB NFDI4Culture team including NFDI4Culture TA1 'Semantic annotation for 3D cultural artefacts MVP', and others.


\end{enumerate}
\begin{enumerate}[start=2]
\item Scope: Review the proposed steps with the community.

\begin{enumerate}
\item Select enhanced publishing features to test.


\end{enumerate}
\begin{enumerate}[start=2]
\item Select architectural computational objects to include.


\item Define uses cases and personas (users). The use cases will from NFDI4Culture architectural digital heritage projects, and from X-Sketchbook developed with the Bartlett School and TIB Open Science Lab.   


\item Define workflow for publishing infrastructures in use at OBP, and at the ADA Pipeline.


\item Write an initial blogpost outlining project and possibilities of and issues in computational publishing for monograph publishers (COPIM).


\item Write a blogpost on 'What is a Computational Book' (COPIM).


\end{enumerate}

\item Proof-of-concept demo:

\begin{enumerate}
\item Execute part one 'Scope' in the OBP and ADA publishing infrastructures and produce a demonstration of the workflows and example publication outputs. The contents would be technical proof-of-concept examples for demonstration purposes. 


\item The proof-of-concept would be run as a public-facing open demonstration for the purpose of community engagement.


\item Write blogpost on the role of the publisher in Computational Publishing (COPIM)


\end{enumerate}

\item Demonstration mockups:

\begin{enumerate}
\item Bring on board two partner projects, the Bartlett and NFDI4Culture from architecture to create demonstration mockups with real publication content.

\begin{enumerate}
\item One mockup would be for contemporary architecture – the Bartlett


\item One mockup for historical architecture – NFDI4Culture


\item Write blogpost reflecting on experiences of authors/communities around Computational Publishing (COPIM)


\item Either organise a stand alone workshop or as part of an NFDI4Culture or COPIM bigger workshop


\end{enumerate}

\end{enumerate}

\end{enumerate}

\subsection{Schedule}\label{H8614052}



\textbf{April/May 2022}

\begin{itemize}
\item Finalise work plan



\textbf{June/July 2022}


\end{itemize}
\begin{itemize}
\item Finalise scope and use cases


\end{itemize}

\textbf{September 2022}

\begin{itemize}
\item Produce Proof-of-concept publication


\end{itemize}

\textbf{Oct / Nov / Dec 2022}

\begin{itemize}
\item Demonstrations; Organise or participate in workshop


\end{itemize}

\subsection{Communities}\label{H637961}


\begin{itemize}
\item \href{https://www.copim.ac.uk/}{COPIM}


\item \href{https://nfdi4culture.de/index.html}{NFDI4Culture}


\item \href{https://osarch.org/}{Open Source in Architecture}


\item Single Source Publishing Community (\href{https://singlesource.pub/}{SSPC})


\item \href{https://nfdi4culture.de/}{NFDI4Culture}


\item \href{https://www.ucl.ac.uk/bartlett/}{The Bartlett}


\item \href{https://curvenote.com/}{Curvenote}


\item \href{https://jupyter.org/}{Jupyter Notebook }


\item \href{https://jupyterbook.org/en/stable/intro.html}{Jupyter Book}


\end{itemize}

\subsection{Planned outputs}\label{H6024753}


\begin{itemize}
\item Project outline: Work plan


\item Blogposts: Announce, and; others


\item Public engagement using mockup building: Proof-of-concept; Demonstrations 


\item Preprint and paper: Wikiversity


\item Workshop (participation and planning) NFDI4Culure – TIB or NFDI context Winter 2022


\item Paper: \href{https://joss.theoj.org/}{The Journal of Open Source Software}, or \href{https://jose.theoj.org/about}{Journal of Open Source Education}


\end{itemize}

\printbibliography[title={Bibliography}]
\end{document}
